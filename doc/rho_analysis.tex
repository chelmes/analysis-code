\documentclass[12pt,a4paper]{scrartcl}
\usepackage[english]{babel}
\usepackage{feynmf}
\usepackage{braket}
\usepackage{amssymb,amsmath}
\usepackage{dsfont}
\usepackage{rotating}
\usepackage{graphicx}
%\usepackage{gensymb}
\usepackage{epsfig}
\usepackage{subfig}
\usepackage{wrapfig}
\usepackage{nicefrac}
\usepackage{color}
\usepackage{float}
\usepackage{textcomp}
\usepackage[utf8]{inputenc}
\usepackage{amssymb,amsmath}
\title{$\rho$admap to Analysis}
\author{Christopher Helmes}
\begin{document}
\maketitle
\section{Prerequisities}
\subsection{Real World}
In principal we want to analyse the mass $m_{\rho}$ and the decay width
$\Gamma_{\rho}$ of the $\rho$-resonance. For this we utilize in a first approach
a $2\times 2$ Correlation Matrix in the center of mass frame and two different
moving frames. The entries of the correlation matrix involve the two different
For each frame under consideration the analysis follows the same steps
operators specified in the paper by Feng, Jansen and Renner: 
\begin{align}
  &\mathcal{O}_{\pi\pi}(t) =
  \pi^+(\vec{P},t)\pi^-(0,t)-\pi^+(0,t)\pi^+(\vec{P},t) \\
  &\mathcal{O}_{\rho}(t) = \rho^0(\vec{P},t)
\end{align}

For each frame under consideration the analysis follows the same steps. The
scattering amplitude in inelastically interacting systems is given by the
relativistic Breit-Wigner formula:
\begin{displaymath}
  a_l = \frac{-\sqrt(s)\Gamma_R(s)}{s-M_R^2+i\sqrt{s}} 
\end{displaymath}
with the CM energy $E_{CM}^2 = s$ and the mass of the resonance $M_R$. The
relation between scattering amplitude and scattering phaseshift $\delta_l$
yields:
\begin{align}
  \tan \delta_l = \frac{\sqrt{s}\Gamma_R(s)}{M_R^2-s}
\end{align}
Taking into account p-wave scattering ($l=1$), the effective coupling
constant $g_{\rho\pi\pi}$, the $\rho$-mass, the $\pi$-mass $m_{\pi}$ and the
center of mass energy give rise to the effective range formula
\begin{displaymath}
  \tan \delta_1 = \frac{g^2_{\rho\pi\pi}}{6\pi}
  \frac{p^3}{E_{CM}(m^2_{\rho}-E^2_{CM})} \quad, \qquad p =
  \sqrt{\frac{E^2_{CM}}{4}-m^2_{\pi}}
\end{displaymath}

\subsection{Lattice}
\section{Analysis}
Given the contractions have been made the resulting Correlation functions
involving every combination of the operators $\mathcal{O}_{\pi\pi}$ and
$\mathcal{O}_{\rho}$ are combined in a bootstrap analysis to extract the two
energy-levels from the correlation matrix. In a first step the $N_C$
configurations are bootstrapped according to the following pattern:
View the (symmetrized) correlation functions as a 2d array in $n_c =
1,\dots,N_C$ and $t = 1,\dots, \nicefrac{T}{2}$, where $T$ is the lattice time
extent. One obtains $N_{BS}$ Bootstrapsamples of the $N_C$ correlation functions at
every value for $t$ as a sum over $N_C$ arbitrarily chosen values of $C(t)$ the
sample with $n_{BS} = 1$ is taken as the sum (average) of the original dataset.
In the end there exists a 2d array of size $N_{BS}\times \nicefrac{T}{2}$ 
A short illustration may clarify things.
%\begin{bmatrix}
%  
%\end{bmatrix}

On every bootstrap sample the correlation matrix $C_{ij}(t) =
\braket{O_i(t)O^{\dag}_j(0)}$ can be built. Cast into a generalized eigenvalue
problem this yields:

\begin{align}
  C(t)v_n(t,t_0) = \begin{pmatrix}
    \braket{\mathcal{O}_{\pi\pi}\mathcal{O}^{\dag}_{\pi\pi}}(t) &
    \braket{\mathcal{O}_{\pi\pi}\mathcal{O}^{\dag}_{\rho}}(t) \\
    \braket{\mathcal{O}_{\rho}\mathcal{O}^{\dag}_{\pi\pi}}(t) &
    \braket{\mathcal{O}_{\rho}\mathcal{O}^{\dag}_{\rho}}(t)
  \end{pmatrix} \begin{pmatrix} v_{n,1}(t,t_0) \\ v_{n,2}(t,t_0) \end{pmatrix} =
  \lambda_n(t,t_0) C(t_0) v_n(t,t_0)
\end{align}
The eigenvalues $\lambda_n$ are given by:
\begin{align}
  \lambda_n(t,t_0) = \exp[-E_n(t-t_0)]
\end{align}
corresponding to the energies of interest (in our case $n < 2$). To prevent
contamination stemming from states with $n > 2$ we use the correlation matrix
$R(t,t_R)$ composed of:
\begin{align}
  R(t,t_R)=C_{2\times 2}(t)C^{-1}_{2\times 2}(t_R)
\end{align}
In this approach the eigenvalues behave as
\begin{align}
  \tilde{\lambda}_n(t) = A_n \cosh\left[-E_n\left(t-\frac{T}{2}\right)\right]
\end{align}


\end{document}

